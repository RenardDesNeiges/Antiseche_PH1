
\begin{multicols}{2}

    \subsection{Systèmes de coordonnées}
    
    \subsubsection*{Cylindriques}
    On représente un point $\vec{M}$ par les trois coordonnées $\rho$, $\varphi$ and $z$:
    \begin{align*}
        \vec{M} = (\rho, \varphi, z).
    \end{align*}
    Les vecteurs unitaires sont représentés en base cartésienne comme:
    \begin{align*}
        \vec{e_\rho} = \begin{pmatrix} 
            \cos \varphi \\
            \sin \varphi \\
            0
        \end{pmatrix}, & 
        \vec{e_\rho} = \begin{pmatrix} 
            - \sin \varphi \\
            \cos \varphi \\
            0
        \end{pmatrix}, & 
        \vec{e_\rho} = \begin{pmatrix} 
            0 \\
            0 \\
            1
        \end{pmatrix}.
    \end{align*}
    On pose les coordonnées d'un point (et leur dérivées) comme suit: 
    \begin{align*}
        \vec{r} = \rho \vec{e_\rho} + z \vec{e_z}, \\
        \vec{v} = \dot{\vec{r}} = \dot{\rho} \vec{e_\rho} + \rho \dot{\varphi} \vec{e_\varphi} + \dot{z} \vec{e_z}, \\
        \vec{a} = \dot{\vec{v}} = \ddot{\vec{r}}, \\
        \ddot{\vec{r}} = (\ddot{\rho} - \rho \dot{\phi}^2) \vec{e_\rho} + (\rho \ddot{\varphi} + 2 \dot{\rho} \dot{\varphi}) \vec{e_\varphi} + \ddot{z} \vec{e_z}.
    \end{align*}
    
    \subsubsection*{Sphériques}
    
    On représente un point $\vec{M}$ par les trois coordonnées $r$, $\theta$ and $\varphi$:
    \begin{align*}
        \vec{M} = (r, \theta, \varphi).
    \end{align*}
    Les vecteurs unitaires sont représentés en base cartésienne comme:
    \begin{align*}
        \vec{e_r} = \begin{pmatrix} 
            \sin \theta \cos \varphi  \\
            \sin \theta \sin \varphi \\
            \cos \theta
        \end{pmatrix}, & 
        \vec{e_\theta} = \begin{pmatrix} 
            \cos \varphi \cos \varphi \\
            \cos \varphi \sin \varphi \\
            - \sin \theta
        \end{pmatrix}, \\
        \vec{e_\varphi} = \begin{pmatrix} 
            -\sin \varphi \\
            \cos \varphi \\
            0
        \end{pmatrix}.
    \end{align*}
    On pose les coordonnées d'un point (et leur dérivées) comme suit:
    \begin{align*}
        \vec{r} = r \vec{e_r}, \\
        \vec{v} = \dot{\vec{r}} = \dot{r} \vec{e_r} + r \dot{\theta} \vec{e_\theta} + \sin \theta \vec{e_\varphi}, \\
        \vec{a} = \dot{\vec{v}} = \ddot{\vec{r}}, \\
        \ddot{\vec{r}} = (\ddot{r} - r \dot{\theta}^2 - r \dot{\varphi}^2 \sin^2 \theta) \vec{e_r}  \\
        + (r \ddot{\theta} + 2 \dot{r} \dot{\theta} - r \dot{\varphi}^2 \sin \theta \cos \theta) \vec{e_\theta} \\
        + (r \ddot{\varphi} \sin \theta + 2 \dot{r} \dot{\varphi}\sin\theta + 2 r \dot{\varphi} \dot{\theta} \cos \theta) \vec{e_\varphi}.
    \end{align*}
    
    \subsection{Equations différentielles}
    
    En physique 1 on traite des équations différentielles linéaire du premier et du second ordre. \textbf{La résolution de ces équations revient à réécrire les équations du mouvement sous une forme dont on connait déjà la solution}. 
    
    
    
    \subsubsection*{Equations différentielles du premier ordre}
    
\end{multicols}
    