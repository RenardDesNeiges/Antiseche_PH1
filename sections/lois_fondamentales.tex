
% \begin{multicols}{2}

    \subsection{Lois de Newton}

    \subsubsection*{1ère loi} \textit{Tout corps persévère dans l'état de repos ou de mouvement uniforme en ligne droite à moins qu'une force agisse sur lui.}
    \[ \vec{F} = 0 \Rightarrow \ddot{x} = 0 \Rightarrow \dot{x} \text{ cst} \]
    
    \subsubsection*{2ème loi} \textit{Les changements de mouvements sont proportionnels à la force motrice, et se font dans la ligne droite dans laquelle cette force est appliquée.}
    \[ \vec{F} = \sum_i \vec{F_i} = \frac{d\vec{p}}{dt }\]
    \textbf{Note: } $\frac{d\vec{p}}{dt} = m\vec{a}$ si $\vec{a}$ est conservé.
    \subsubsection*{3ème loi}
    \[ \vec{F}^{2 \rightarrow 1} + \vec{F}^{1 \rightarrow 2} = 0\]
    \textit{La somme des forces internes entre 2 corps prise "des deux côtés" est toujours nulle. A toute action est opposée une réaction égale.}
    \\
    Ce qu'il faut en retirer c'est que les forces internes d'un système sont sans influence sur le comportement du centre de masse du système.
    
    % \end{multicols}
    
    \newpage