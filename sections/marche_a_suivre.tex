
\begin{multicols}{2}
    \subsection{Choisir quelle méthode de résolution utiliser}
    Si on a besoin de calculer des equations du mouvement en fonction du temps, ou des frottements $\Rightarrow$ \textit{2ème loi}. Si on s'intéresse à des situations au cours du temps (choc, chute etc.) on peux passer par la conservation de l'énergie mécanique.
    \subsection{Quand on traite le problème avec la 2ème loi}
    \begin{enumerate}
        \item Lire la donnée (attentivement)
        \item Mettre en emphase ce qu'on recherche
        \item Choisir un repère de coordonnées adapté au problème (selon les symétries, pour faciliter la projection, galiléen   ou non)
        \item Faire un dessin
        \begin{enumerate}
            \item Dessiner les vecteurs unitaires du repère
            \item Dessiner les forces
        \end{enumerate}
        \item Poser les contraintes du système (qu'est-ce qui est constant?)
        \item Exprimer les positions, vitesses, accelerations dans le repère
        \item Exprimer les forces dans le repère
        \begin{enumerate}
            \item S'il y en a, calculer les forces d'inertie et les ajouter au dessin
        \end{enumerate}
        \item Poser la deuxième loi de newton 
        \[\sum \vec{F}_\text{i} = m \cdot \vec{a}\]
        \item Eventuellement, résoudre l'équation (si demandé par la donnée)
        \begin{enumerate}
            \item Relire la donnée avant de résoudre
            \item Hypothèses d'échelle, regarder si on peut négliger qqch...
        \end{enumerate}
    \end{enumerate}
    \subsection{Quand on traite le problème avec la conservation de l'énergie}
    \begin{enumerate}
        \item Lire la donnée (attentivement)
        \item Mettre en emphase ce qu'on recherche
        \item Choisir un repère de coordonnées adapté au problème (selon les symétries, pour faciliter la projection, galiléen   ou non)
        \item Faire un dessin
        \begin{enumerate}
            \item Dessiner les vecteurs unitaires du repère
            \item Dessiner les forces
        \end{enumerate}
        \item Poser les contraintes du système (qu'est-ce qui est constant?)
        \item Lister les situations pertinentes (i.e. "avant la chute", "après la chute, avant le choc", "après le choc" etc.) et les numéroter.
        \item Choisir le $0$ pour l'énergie potentielle
        \item Faire le bilan des énergies mécaniques et des quantités de mouvements (si choc) à chaque situation, noter clairement ce qui est égal à 0
        \item Résoudre les équation (relire la donnée)
    \end{enumerate}
    
\end{multicols}