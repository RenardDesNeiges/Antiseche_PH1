
% \begin{multicols}{2}
    \subsection{Quand on traite le problème avec la 2ème loi}
    \begin{enumerate}
        \item Lire la donnée (attentivement)
        \item Mettre en emphase ce qu'on recherche
        \item Choisir un repère de coordonnées adapté au problème (selon les symétries, pour faciliter la projection, galiléen   ou non)
        \item Faire un dessin
        \begin{enumerate}
            \item Dessiner les vecteurs unitaires du repère
            \item Dessiner les forces
        \end{enumerate}
        \item Poser les contraintes du système (qu'est-ce qui est constant?)
        \item Exprimer les positions, vitesses, accelerations dans le repère
        \item Exprimer les forces dans le repère
        \begin{enumerate}
            \item S'il y en a, calculer les forces d'inertie et les ajouter au dessin
        \end{enumerate}
        \item Poser la deuxième loi de newton 
        \[\sum \vec{F}_\text{i} = m \cdot \vec{a}\]
        \item Eventuellement, résoudre l'équation (si demandé par la donnée)
        \begin{enumerate}
            \item Relire la donnée avant de résoudre
            \item Hypothèses d'échelle, regarder si on peut négliger qqch...
        \end{enumerate}
    \end{enumerate}
    
    % \end{multicols}