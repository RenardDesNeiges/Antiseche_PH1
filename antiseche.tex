\documentclass[11pt]{article}

\usepackage{sectsty}
\usepackage{graphicx}
\usepackage[T1]{fontenc}
\usepackage{epigraph} %quotes
\usepackage{amssymb} %math symbols
\usepackage{mathtools} %more math stuff
\usepackage{amsthm} %theorems, proofs and lemmas
\usepackage[ruled,vlined]{algorithm2e} %algoritms/pseudocode

\usepackage[french]{babel}

%% Theorem notation
\newtheorem{theorem}{Theorem}[section]
\newtheorem{corollary}{Corollary}[theorem]
\newtheorem{lemma}[theorem]{Lemma}
\newtheorem{problem}{Problem}[section]

%% declaring abs so that it works nicely
\DeclarePairedDelimiter\abs{\lvert}{\rvert}%
\DeclarePairedDelimiter\norm{\lVert}{\rVert}%

% Swap the definition of \abs* and \norm*, so that \abs
% and \norm resizes the size of the brackets, and the 
% starred version does not.
\makeatletter
\let\oldabs\abs
\def\abs{\@ifstar{\oldabs}{\oldabs*}}
%
\let\oldnorm\norm
\def\norm{\@ifstar{\oldnorm}{\oldnorm*}}
\makeatother

% Marges
\topmargin=-0.45in
\evensidemargin=0in
\oddsidemargin=0in
\textwidth=5.5in
\textheight=9.0in
\headsep=0.5in


\title{Antisèche pour le cours de mécanique newtonienne}
\date{\today}
\author{Titouan Renard}

\begin{document}
\maketitle	


\section{Marche à suivre pour résoudre un exercice}

\begin{enumerate}
    \item Lire la donnée (attentivement)
    \item Mettre en emphase ce qu'on recherche
    \item Faire un dessin
    \begin{itemize}
        \item Dessiner les forces
    \end{itemize}
    \item Choisir un repère de coordonnées adapté au problème (selon les symétries, pour faciliter la projection, euclidien ou non)
    \item Poser les contraintes du système (qu'est-ce qui est constant?)
    \item Exprimer les positions, vitesses, accelerations dans le repère
    \item Exprimer les forces dans le repère
    \begin{itemize}
        \item Exprimer s'il y en a les forces d'inertie
        \item Relire la donnée avant de résoudre
    \end{itemize}
    \item Poser la deuxième loi de newton ($\sum \vec{F}_\text{ext} = m \cdot \vec{a}$)
    \begin{itemize}
        \item Hypothèses d'échelle, regarder si on peut négliger qqch...
    \end{itemize}
    \item Eventuellement, résoudre l'équation (si demandé par la donnée)
\end{enumerate}

\end{document}